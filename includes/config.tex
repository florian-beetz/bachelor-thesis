%===============================================================================
% Zentrale Layout-Angaben und Befehle
%===============================================================================
%
% Für bessere Sicht von falschen Umbrüchen die Option draft benutzen.
% Dadurch können aber die eingebundenen Bilder nicht sichtbar sein.
\documentclass[a4paper, 12pt]{article}
%
% Hier zunächst die benötigten Packages
\usepackage[utf8]{inputenc}
\usepackage{fancyhdr}
\usepackage[T1]{fontenc}
\usepackage{ae}
\usepackage{listings}
\usepackage{color}
\usepackage{listings}
\usepackage{wrapfig}
\usepackage[printonlyused]{acronym}
\usepackage{url}
\usepackage{hyperref}
\usepackage{longtable, booktabs}

\usepackage[english]{babel}
\usepackage{csquotes}
\usepackage[backend=biber,style=numeric]{biblatex}

%
% Einbindung des Grafik-Pakets
\ifx\pdfoutput\undefined
	\usepackage[dvips]{graphicx}
\else
	\usepackage[pdftex]{graphicx}
\pdfcompresslevel=9
\pdfpageheight=297mm
\pdfpagewidth=210mm
\fi

\usepackage{listings}
\usepackage{color}

\definecolor{backgroundshade}{RGB}{248,248,248}
\definecolor{keywordcolor}{rgb}{0.13,0.29,0.53}
\definecolor{commentcolor}{rgb}{0.56,0.35,0.01}
\definecolor{numbercolor}{rgb}{0.00,0.00,0.81}
\definecolor{stringcolor}{rgb}{0.31,0.60,0.02}

\lstset{ 
  backgroundcolor=\color{backgroundshade},   % choose the background color; you must add \usepackage{color} or \usepackage{xcolor}; should come as last argument
  breakatwhitespace=false,         % sets if automatic breaks should only happen at whitespace
  breaklines=true,                 % sets automatic line breaking
  captionpos=b,                    % sets the caption-position to bottom
  extendedchars=true,              % lets you use non-ASCII characters; for 8-bits encodings only, does not work with UTF-8
  keepspaces=true,                 % keeps spaces in text, useful for keeping indentation of code (possibly needs columns=flexible)
  language=Java,
  basicstyle=\small\ttfamily,        % the size of the fonts that are used for the code
  commentstyle=\itshape\color{commentcolor},    % comment style
  emphstyle=\color{keywordcolor},
  keywordstyle=\color{keywordcolor},       % keyword style
  stringstyle=\color{stringcolor},     % string literal style
  emph={open,module,requires,opens,to,transitive,provides,uses,with},            % if you want to add more keywords to the set
  showspaces=false,                % show spaces everywhere adding particular underscores; it overrides 'showstringspaces'
  showstringspaces=false,          % underline spaces within strings only
  showtabs=false,                  % show tabs within strings adding particular underscores
  stepnumber=0,                    % the step between two line-numbers. If it's 1, each line will be numbered
  tabsize=4,	                   % sets default tabsize to 2 spaces
}

%
% Page-Layout
\setlength\headheight{14pt}
\setlength\topmargin{-15,4mm}
\setlength\oddsidemargin{-0,4mm}
\setlength\evensidemargin{-0,4mm}
\setlength\textwidth{160mm}
\setlength\textheight{252mm}
%
% Absatzeinstellungen
\setlength\parindent{0mm}
\setlength\parskip{2ex}
%
% Kopf- und Fusszeile
\pagestyle{fancy}
\fancyhf{} % alles löschen
\fancyhead[LO]{\footnotesize\sc\nouppercase{\leftmark}}
\fancyfoot[LO]{\footnotesize\sc Lehrstuhl für Praktische Informatik}
\fancyfoot[RO]{\thepage}
\renewcommand{\headrulewidth}{0pt}
\renewcommand{\footrulewidth}{0pt}
%
% Bessere Fehlermeldungen
\errorcontextlines=999
%
% Anweisung zur Erstellung der Titelseite
% #1 Bachelorarbeit || Masterarbeit
% #2 = Studiengang
% #3 = Titel der Arbeit
% #4 = Autor
% #5 = Abgabedatum
\renewcommand{\maketitle}[5]
{
\pagenumbering{Alph}
\begin{titlepage}
\centering
\begin{minipage}[t]{16cm}
\begin{minipage}{3cm}
    \includegraphics[height=26mm]{includes/vs-logo}
\end{minipage}
\hfill
\begin{minipage}{9cm}
  \centering
    Otto-Friedrich-Universität Bamberg\\[12pt]
    {\Large Lehrstuhl für Praktische Informatik}
\end{minipage}
\hfill
\begin{minipage}{3cm}
    \includegraphics[height=26mm]{includes/UB-Logo-neu_blau-cmyk}
\end{minipage}
\end{minipage}\\[130pt]
{\LARGE #1}\\[24pt]
im Studiengang #2\\
der Fakultät Wirtschaftsinformatik und Angewandte Informatik\\
der Otto-Friedrich-Universität Bamberg\\[90pt]
Zum Thema:\\[24pt]
{\Huge #3}\\[60pt]
\vfill
\begin{minipage}{\textwidth}
\center
Vorgelegt von:\\
{\Large #4\\[12pt]}
Themensteller:\\
Prof. Dr. Guido Wirtz\\[12pt]
Abgabedatum:\\
#5\\
\end{minipage}
\end{titlepage}
}
%
% Anweisung zur Erstellung der Eigenständigkeitserklärung
% #1 = Typ der Arbeit
% #2 = Datum
% #3 = Vorname Name
\newcommand{\makedeclaration}[3]
{
	\fancyhead[LO]{\footnotesize\sc\nouppercase{Eigenständigkeitserklärung}}
	%
			\vspace*{18cm}
			Ich erkläre hiermit gemäß §17 Abs. 2 APO, dass ich die vorstehende #1 selbstständig verfasst 
			und keine anderen als die angegebenen Quellen und Hilfsmittel verwendet habe.\\
			\vspace*{1cm}
			
			Bamberg, den #2 \hspace{5cm} #3
	%
}
%
% Wird für Hintergrund von Codelistings benötigt
\definecolor{hellgrau}{gray}{0.9}
%
% Einstellungen für Java-Code
\lstdefinestyle{javaStyle}{%
  basicstyle=\small,%
  backgroundcolor=\color{hellgrau},%
  keywordstyle=\bfseries,%
  showstringspaces=false,%
  language=Java,%
  numbers=left,%
  numberstyle=\tiny,%
  stepnumber=1,%
  numbersep=5pt,%
  extendedchars=true,%
  xleftmargin=2em,%
  lineskip=-1pt,%
  breaklines%
}
%
% neues environment für Java-Sourcecode
% #1 = "caption={Hier eigene Überschrift}, label={Hier eigenes Label}"
\lstnewenvironment{javacode}[1][]{%
\lstset{style=javaStyle,#1}%
}{}
%
% Befehl zum Einbinden von Java-Sourcecode aus Datei
% #1 = Dateiname relativ zu src-Verzeichnis
% #2 = Überschrift
% #3 = Label
\newcommand{\javafile}[3]{%
   \lstinputlisting[%
     caption={#2},%
     label={#3},%
     style=javaStyle]{src/#1}%
}
%
% Einbindung eines Bildes
% #1 = label für \ref-Verweise
% #2 = Name des Bildes ohne Endung relativ zu images-Verzeichnis
% #3 = Beschriftung
% #4 = Breite des Bildes im Dokument in cm
\newcommand{\bild}[4]{%
  \begin{figure}[htb]%
    \begin{center}%
      \includegraphics[width=#4cm]{images/#2}%
      \vskip -0.3cm%
      \caption{#3}%
      \vskip -0,2cm%
      \label{#1}%
    \end{center}%
  \end{figure}%
}
%
% Umgebung für Fliesstext um Grafik
% #1 = Ausrichtung: r, l, i, ...
% #2 = Breite des Bildes in cm
% #3 = Name des Bildes ohne Endung relativ zu images-Verzeichnis
% #4 = Beschriftung
% #5 = label für \ref-Verweise
\newcommand{\fliesstext}[5]{%
\begin{wrapfigure}{#1}{#2cm}%
\includegraphics[width=#2cm]{images/#3}%
\caption{#4}%
\label{#5}%
\end{wrapfigure}%
}
%%% Local Variables:
%%% mode: latex
%%% TeX-master: t
%%% End:
